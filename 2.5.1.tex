\documentclass[12pt,a4paper]{article}
\usepackage[utf8]{inputenc}
\usepackage[russian]{babel}
\usepackage[OT1]{fontenc}
\usepackage{graphicx}
\usepackage[left=1cm,right=1cm,top=1cm,bottom=1cm]{geometry}
\author{Владимир Журавлев}
\usepackage{pgfplots}
\usepackage{amsmath}
\pgfplotsset{compat=1.9}
\pagestyle{plain}
\usepackage{pgfplotstable,filecontents}

\begin{document}
\begin{flushright}
Работу выполнил:\\
\textbf{Журавлев Владимир, 621 гр.\\}
16.03 - 23.03

\end{flushright}
\begin{center}
\begin{LARGE}

\vspace{\baselineskip}
Лабораторная работа №2.5.1\\
\textbf{Измерение коэффициента поверхностного натяжения жидкости}\\
\vspace{\baselineskip}

\end{LARGE}
\end{center}

\noindent\textbf{Цель работы:} 
\begin{itemize}
\item Измерение коэффициента поверхностного натяжения исследуемой жидкости с использованием известного коэффициента поверхностного коэффициента поверхностного натяжения другой жидкости
\item Определение полной поверхностной энергии и теплоты, необходимой для изотермического образования единицы поверхности жидкости.
\end{itemize}
\textbf{Оборудование:} прибор Ребиндера с термостатом; исследуемые жидкости; стаканы\\
\begin{Large}
\begin{center}
\textbf{1. Теория}\\
\end{center}
\end{Large}
\paragraph{1. Формула Лапласа:} Под искривленной поверхностью жидкости, то по разные стороны от поверхностного слоя существует разность давлений $\Delta P$, пропорциональная кривизне поверхности:
\begin{equation}
\Delta p = \sigma  \left[   \frac{1}{r_{1}}+\frac{1}{r_{2}} \right]
\end{equation}
Если поверхностных слоев несколько, то следует умножить разность давлений на их количество.
\paragraph{2. Полная поверхностная энергия:}
\begin{equation}
U_{s} = \left( \sigma - T\frac{d \sigma}{dT} \right) S
\end{equation}
\paragraph{3. Изотермическое образование единицы площади пленки:}
Из первого начала термодинамики: 
\begin{equation}
\delta Q = d U - \sigma d S = \left( - T \frac{d \sigma}{dT} \right)  dS \\
\end{equation}
\[q = - T \frac{d \sigma}{dT}\]
\newpage
\begin{Large}
\begin{center}
\textbf{2. Ход работы}\\
\end{center}
\end{Large}
\paragraph{1. Определение радиуса иглы}

Сначала на известной жидкости при комнатной температуре определим внутренний радиус иглы. Для этого необходимо опустить иглу чуть ниже поверхности жидкости, так, чтобы гидростатическое давление оставалось малым. Затем добьемся медленного роста давления и определим максимальное, при котором воздух выходит из иглы.
\begin{center}
\begin{large}
\begin{tabular}{ccc|cc}
 
  $\Delta P$, мм сп. ст.& $\Delta P$, Па & $\sigma$, н/м & $d$, мм \\ 
  \hline 
  $53 \cdot 0.2$& $84.2$ & $23 \cdot 10^{-3}$ & $1.10$ \\ 
  \end{tabular}   
\end{large}
\end{center}
Погрешность метода: 
\[\frac{\Delta d}{d} = \frac{\Delta P}{P}  = \frac{3}{53} \approx 0.06\]
Откуда:
{\large \[ d = 1.10 \pm 0.07 \text{ мм} \]}


Теперь измерим диаметр иглы с помощью микроскопа:\\
{\large \[d = 1.00 \pm 0.05 \text{ мм} \]}

\paragraph{2. Определение коэффициента поверхностного натяжения}


Тщательно промоем иглу, затем поместим ее в воду, и измерим $h_1 = 7 \text{ мм}$. Максимальное давление в этом случае: $\Delta P  =  133 \cdot 0.2 \text{ мм. сп. ст.}$\\
Опустим иглу до упора, $h_1 = 22 \text{ мм}$, $\Delta h = 15 \text{ мм}$\\
При этом разность давлений по манометру: $\Delta P = 14.9  \text{ мм вод. ст.} $\\

Теперь снимем зависимость $\sigma (T)$: будем медленно нагревать воду и снимать точки:\\
\begin{center}

\begin{large}
\begin{tabular}{ccc|cc}
\hline 
$t^{\circ}, C$ & $T$, K & $L$, мм & $\Delta P$ & $\sigma, \; \frac{\text{Н}}{\text{м}}$ \\ 
\hline 
22&295.2&225&220.2345&0.055\\
28&301.2&223&217.05606&0.054\\
32&305.2&221&213.87762&0.0534\\
38.7&311.9&219&210.69918&0.052\\
43.8&317&217&207.52074&0.0518\\
49.7&322.9&215&204.3423&0.0510\\
54.7&327.9&213&201.16386&0.0502\\
60&333.2&212&199.57464&0.049\\

\end{tabular} 
\end{large}
\end{center}
Теперь мы можем построить график $\sigma (T)$, вычислить МНК и определить $\frac{d \sigma}{dT}$\\
\paragraph{3. определение полной поверхностной энергии и теплоты}:\\
Полная поверхностная энергия и теплота изотермического образования дается выражением: 
\[U = S\left( \sigma - T \frac{d \sigma}{dT}   \right) \]
тогда на единицу площади в изотермическом процессе подведенное тепло равно 
\[q = - T \frac{d \sigma}{dT}\]
Вычислив их для каждой точки построим график:\\
\newpage
\begin{figure}[h]
\centering
\begin{tikzpicture}
	\begin{axis}[
	 scale=1.58,
	  xlabel={{\large $T, K$}}, 
	  ylabel={{\large $\sigma, \; \frac{\text{Н}}{\text{М}}$}},
				  xmin=290,xmax=345,ymin=0.048,ymax=0.12]
		\addplot[color=blue,  mark=*, only marks] table[col sep=comma] {plot1.csv};
		\addplot[color=black] coordinates {
		(0, 0.0975)
		(400, 0.04)
		};
		\addplot[color=red] coordinates {
		(0, 0)
		(400, 0.06)
		};
		\addplot[color = gray] coordinates {
		(0, 0.099338625)
		(1000, 0.099444015)
		};
		\legend{Нагревание, МНК, q, $U/S$}
	\end{axis}
\end{tikzpicture}
\caption{График зависимости $\sigma $ от $T$ \label{plot}}
\end{figure}
\begin{Large}
\begin{center}
\textbf{3. Результат}\\
\end{center}
\end{Large}
\paragraph{1. Погрешности}:\\
Приборные погрешности:
\[\varepsilon_{man} \simeq  0.015  \]
\[\varepsilon_{line} \simeq  0.035  \]
\[\varepsilon_{term} \simeq  0.02 \]}
МНК:
\[\varepsilon_{OLS} \simeq  0.02 \]
\paragraph{2. Полученные величины}:\\
\begin{Large}
\[\boxed{ \frac{d \sigma}{dT} = - 1.5 \cdot 10^{-4} \pm  7 \cdot 10^{-6}  }\]
\[\boxed{ \frac{U}{S}  = 0.099 \pm 0.004 }\]
\end{Large}
Значения $\sigma (T)$ представлены в таблице. Погрешность $\frac{\Delta \sigma}{\sigma} \simeq 0.04$ 
\end{document}